\textbf{Resumen}:

Resumen ES.

\textbf{Palabras claves}: clave1, clave2, ..., clave5

\vspace{1cm}
\begin{center}
  \rule{0.5\textwidth}{.4pt}
\end{center}
\vspace{1cm}

\textbf{Abstract}:

Benchmarks play a crucial role in developing, researching, and evaluating models for practical use. Existing benchmarks and frameworks have limitations. Different benchmarks partially compensate for each other, but none is universal: issues include saturation, data leakage, limited metrics and outputs, and high energy and computational costs.



We propose an approach to extract the maximum useful information from each benchmark run, enhancing insights for researchers and practitioners while reducing environmental impact. By making benchmark executions more informative, redundant runs can be avoided, lowering energy consumption. The approach enables flexible task selection, fine-grained configuration, and integration of multiple evaluation criteria in a single execution. A prototype was implemented to demonstrate feasibility, validate functionality, and examine limitations.




\textbf{Keywords}: keyword1, keyword2, ..., keyword5
